\documentclass[12pt]{article}

\title{Precise Acquisition LIBS Movement Software: An Easily Usable Control Software for Robotized Optomechanical Systems}
\author{
    Gideon Shaked\\
    \textbf{Presenting Author}
    \and
    Max Vallone\\
    \textbf{Non-presenting Author} 
    \and
    Robert Dubard\\
    \textbf{Non-presenting Author}
    \and
    Claudia Ochatt\\
    \textbf{Project Lead}
}
\date{\today}

\usepackage{csquotes}

\begin{document}
    \maketitle

    \begin{itemize}
        \item \textbf{Novel Aspect:} In-house development of flexible and user-friendly control software for robotized optomechanical positioning systems
        \item \textbf{Primary Research Category:} LIBS -- Innovative instrumentation
        \item \textbf{Secondary Research Category:} LIBS -- Methodology
    \end{itemize}

    Emerging technologies for robotic positioning of optical components show promise in their ability to deliver improved accuracy and precision. 
    However, 
    they generally lack the foundational software necessary for operation by researchers. 
    Our goal in filling this gap was to 
    create an easily usable control software that maintains the accuracy deliverable by robotic systems. 
    We developed the Precise Acquisition LIBS Movement Software (PALMS), 
    a program designed to control optomechanical robotic systems. 
    We adopted a client-server approach to programming PALMS\@. 
    PALMS consists of a self-contained graphical user interface client that can run on 
    Windows, 
    macOS, 
    and Linux 
    and a server that runs on a Raspberry Pi Zero 
    that is directly connected to the robotized positioning system. 
    PALMS allows the operator to position multiple axes across their ranges of motion. 
    PALMS also features a lock functionality that lets the operator \enquote{lock} all of the axes for as long as desired up to 60 seconds. 
    This enables sample switching 
    and manual movements 
    in close proximity to the robotized positioning system 
    without accidentally affecting the arrangement of the system. 
    The client and the server communicate via TCP sockets to achieve low latency and connection integrity. 
    Because the client and server only need to be on the same local area network to communicate, 
    the operator can remotely control the system and monitor it via networked cameras. 
    As such, people who were previously unable to operate or observe the operation of a positioning system 
    due to lack of space or inability to wear protective gear 
    can now do so. 
    When used in conjunction with a sample and fiber optic positioning system 
    such as the Five Axis Positioning System (FAPS), 
    PALMS allows the operator to update the desired planar position (XY), 
    effective focal lens distance (Z), 
    angular tilt of the sample relative to the fiber optic (A), 
    and fiber optic position (B). 
    By making the PALMS software open source under the MIT license, 
    researchers can freely use and modify it to fit their specific needs. 
    PALMS opens the door for easily usable software-controlled positioning in the lab.

\end{document}
